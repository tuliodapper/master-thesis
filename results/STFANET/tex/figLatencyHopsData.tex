\begin{figure}[!htb]
\centering
\caption{Latency for data packet transmissions and number of hops.}
\begin{tikzpicture}
    \begin{axis}[
        ylabel={Time [ms]},
        xlabel={Number of Relay Nodes},
        % (adjusting the `ymin' value is just to make it look a little bit better)
        ymin=0,
        % ymax=1e6,
        % there should be no gap between the bars in one group
        ybar=0pt,
        % use data from the table for the xticklabels
        xtick=data,
        xticklabels from table={\simcsvdicLatencyData}{NumberOfNodes},
        % to start the bars from the bottom edge of the plot
        % (otherwise they would start from 10^0
        %  borrowed from <http://tex.stackexchange.com/a/86688/95441)
        % log origin=infty,
        % adjust the size of the bars so they don't overlap
        % (you can play with the numerator to change the gap between the groups)
        bar width=0.18,
        % enlarge the x limits so all of the bars are shown
        % (play with the value to adjust the gap on the sides of the plot)
        enlarge x limits={abs=0.6},
        legend image code/.code={%
            \draw[#1, draw=black] (0cm,-0.05cm) rectangle (0.4cm,0.1cm);
        },  
        legend style={
            draw=black, % ?
            text depth=0pt,
            at={(0.5,1.0)},
            anchor=center,
            legend columns=-1,
            /tikz/every even column/.append style={column sep=0.5cm},
        },]
        % % add `nodes near coords'
        % nodes near coords={
        %     % because internally PGFPlots works with floating point numbers, we
        %     % change them to fixed point numbers
        %     \pgfkeys{
        %         /pgf/fpu=true,
        %         /pgf/fpu/output format=fixed,
        %     }%
        %     % check if numbers are greater than 1000 and if so, divide them by
        %     % 1000 to convert them from ms to s scale
        %     \pgfmathparse{
        %         ifthenelse(
        %             \pgfplotspointmeta < 1000,
        %             \pgfplotspointmeta,
        %             \pgfplotspointmeta/1000
        %         )
        %     }%
        %     % to now decide which of the two cases we have, we compare the
        %     % point meta value, but because `\ifnum' compares integers, we first
        %     % have to convert the fixed number to an integer
        %         \pgfmathtruncatemacro{\Y}{\pgfplotspointmeta}%
        %     \ifnum\Y<1000
        %         \pgfmathprintnumber{\pgfmathresult}\,ms
        %     \else
        %         \pgfmathprintnumber{\pgfmathresult}\,s
        %     \fi
        % },
        % % set the style of the `nodes near coords'
        % nodes near coords style={
        %     font=\tiny,
        %     rotate=90,
        %     anchor=west,
        % },
        % % as basis for the `nodes near coords' use the raw y values
        % point meta=rawy,
        ]
        % add the data rows
        \foreach \i in {1,...,4}{
            \if\i1
                \addplot    +[  ybar,
                                draw=black,
                                fill=white,
                                error bars/.cd,
                                y dir=both,
                                y explicit
                            ] plot
                            table [
                                x expr=\coordindex,
                                y index=1,
                                y error index=2,
                                col sep=comma,
                            ] {\simcsvdicLatencyData};
            \fi
            \if\i2
                \addplot    +[  ybar,
                                draw=black,
                                fill=light-gray,
                                error bars/.cd,
                                y dir=both,
                                y explicit
                            ] plot
                            table [
                                x expr=\coordindex,
                                y index=3,
                                y error index=4,
                                col sep=comma,
                            ] {\simcsvdicLatencyData};
            \fi
            \if\i3
                \addplot    +[  ybar,
                                draw=black,
                                fill=dark-gray,
                                error bars/.cd,
                                y dir=both,
                                y explicit
                            ] plot
                            table [
                                x expr=\coordindex,
                                y index=5,
                                y error index=6,
                                col sep=comma,
                            ] {\simcsvdicLatencyData};
            \fi
            \if\i4
                \addplot    +[  ybar,
                                draw=black,
                                fill=black,
                                error bars/.cd,
                                y dir=both,
                                y explicit
                            ] plot
                            table [
                                x expr=\coordindex,
                                y index=7,
                                y error index=8,
                                col sep=comma,
                            ] {\simcsvdicLatencyData};
            \fi
            
            % add the legend entry
            \addlegendentryexpanded{\i};
        }
\end{axis}
\end{tikzpicture}
\imagesource{\thisauthor}
\label{figLatencyHopsData}
\end{figure}